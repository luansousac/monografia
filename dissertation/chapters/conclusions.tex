\chapter{Conclusões e Trabalhos Futuros} \label{chapter:conclusions}

A popularidade dos modelos lineares deve-se a dois fatores: uma formulação matemática simples e propriedades bastante exploradas. Entretanto, a capacidade de generalização destes modelos é limitada, não sendo eficiente em solucionar muitos problemas do mundo real. A não-linearidade característica de tais problemas requer a existência de modelos não-lineares que sejam capazes de detectar as relações de dependência necessárias para predições bem sucedidas.

O uso de métodos de \textit{kernel} permitiu que diversos modelos não-lineares fossem criados como extensões de modelos lineares já conhecidos. A ideia subjacente é realizar uma transformação espacial no problema, embutindo os padrões em um espaço de Hilbert, onde estes apresentam comportamento linear; dessa forma, é possível o uso de métodos lineares neste novo espaço.

A descoberta de uma função de \textit{kernel} que seja capaz de realizar essa transformação mostra-se uma tarefa árdua, pois requer conhecimento \textit{a priori} sobre o problema a ser solucionado. Este trabalho apresentou uma proposta de criação de funções de \textit{kernel} que possam ser utilizadas em problemas de regressão, denominada \textit{Genetic Kernels for Regression} (GKR), que utiliza uma abordagem baseada em PG. As funções de \textit{kernel} são então utilizadas no método KRR.

O modelo GKR utiliza o arcabouço padrão de PG, onde os indivíduos de uma população possuem representação genética em forma de árvores sintáticas. Ao longo das gerações, cada indivíduo é avaliado de acordo com sua eficiência em resolver o problema em mãos. O valor de aptidão é estimado através de validação cruzada de \textit{k-folds} sobre o uso da função de \textit{kernel} subjacente no método KRR. A seleção dos melhores indivíduos é realizada pelo operador \textit{lexicographic tournament} que procura evitar tanto o \textit{overfitting} quanto crescimento desenfreado da árvore sintática (\textit{bloat growth}). Os operadores de cruzamento e mutação padrões da PG são então aplicados para a criação das gerações seguintes.

Através dos resultados obtidos, tanto para problemas de natureza artificial quanto real, pode-se inferir que o modelo GKR é bastante competitivo com modelos considerados estado-da-arte, como SVRs e RNAs do tipo MLP e RBF. Mesmo com um consumo de tempo relativamente maior, o GKR apresenta boa robustez, pois quando apresentado a diferentes subconjuntos de treinamento, consegue criar funções de \textit{kernel} representativas do problema e que, de certa forma, são equivalentes em termos de desempenho.

Para trabalhos futuros, considera-se o uso de paralelismo para o processo de evolução, utilizando GPUs. Isso pode ser realizado através da abordagem mestre-escravo (do inglês \textit{master-slave}) \cite{oussadeine1997}.