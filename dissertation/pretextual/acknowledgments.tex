\begin{agradecimentos}
Um curso de graduação permite que o aluno seja exposto à uma grande variedade de assuntos. Tive a felicidade de encontrar um tema com o qual me identificasse ao fim do segundo ano de faculdade. Sempre tive um fascínio (de observador) pela área de Inteligência Artificial. A primeira oportunidade de trabalhar nessa área veio em meados de 2016, sob orientação do professor Ajalmar. Desde então, outras oportunidades surgiram.

Agradeço ao professor Ajalmar por me acolher como seu aluno e orientando, sugerir o tema desta monografia como trabalho de conclusão de curso e me orientar durante todos os meses de trabalho. Agradeço ainda pelos conselhos e experiências compartilhados comigo, que me permitiram dar os passos iniciais como pesquisador da área.

A universidade nos permite contato com pessoas de diferentes crenças, costumes e pensamentos. Pessoalmente, acredito que essa seja sua maior virtude. Sem dúvidas, as amizades que fiz durante toda a graduação me ajudaram a concluir este momento da minha vida. Agradeço a todos estes, desde colegas de turma aos amigos que fiz no Laboratório de Visão Computacional e Inteligência Artificial (LabVicia). Dentre todos, quero destacar três grandes amigos que fiz: Davi, que chegou a ser cofundador de uma tentativa de \textit{startup}; Victor, que hoje trabalha comigo e que em muitos momentos foi integrante de equipe nos trabalhos da faculdade; e Iedo, com o qual compartilho (atualmente) uma amizade de mais de uma década.

O processo de escrita de uma monografia é conturbado e diferente de qualquer outro trabalho mais ``comum'' da faculdade. Para encarar as dificuldades e momentos de tristeza/ansiedade, precisamos recorrer e buscar apoio naqueles que nos amam. Agradeço à minha mãe, por ser mãe. Dito isso, pode-se deduzir que esta mulher fez (e faz) de tudo por mim, desde me dar amor e carinho à me fornecer alimento e moradia; acima de tudo, me ensinando a ser uma pessoa respeitosa e responsável. Agradeço à meu pai, que mesmo à distância, sempre esteve próximo de mim, apoiando, incentivando e dando um dos melhores exemplos que posso ter de trabalho duro, caráter e responsabilidade. Agradeço ainda à meu padrasto, que assim como meu pai, sempre me mostrou a importância do trabalho e por ser um companheiro que ama e cuida de minha mãe; o fruto desse amor é o meu irmão, que traz alegria à todos em nossa família, assim como as irmãs que tenho por parte de pai. Destes, faço um último agradecimento à minha madrasta, por ter sido (e ser) uma parte essencial da minha vida.

Neste momento, gostaria de agradecer a pessoa que acompanhou todo o percurso que trilhei neste trabalho: Hannah. Uma das melhores pessoas que conheço e que me apoiou em todos os momentos de dificuldades, celebrando também as vitórias que tive. À você, palavra alguma vai expressar com precisão o que sinto por ti e o quanto sou grato por ter você em minha vida.

Agradeço também aos professores que tive durante toda minha vida, especialmente a partir do ensino médio (até onde minha memória permite-me lembrar) chegando aos da faculdade. Agradeço pelo empenho em ensinar as ferramentas necessárias na formação acadêmica, mas acima disso, pelos ensinamentos sobre a vida, respeito com o próximo e pensamento crítico. Também agradeço aos servidores do IFCE, por manterem a instituição funcionando, sempre com zelo pelo bem público.

Por fim, agradeço à você, leitor, pelo interesse neste trabalho.
\end{agradecimentos}